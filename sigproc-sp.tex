% THIS IS SIGPROC-SP.TEX - VERSION 3.1
% WORKS WITH V3.2SP OF ACM_PROC_ARTICLE-SP.CLS
% APRIL 2009
%
% It is an example file showing how to use the 'acm_proc_article-sp.cls' V3.2SP
% LaTeX2e document class file for Conference Proceedings submissions.
% ----------------------------------------------------------------------------------------------------------------
% This .tex file (and associated .cls V3.2SP) *DOES NOT* produce:
%       1) The Permission Statement
%       2) The Conference (location) Info information
%       3) The Copyright Line with ACM data
%       4) Page numbering
% ---------------------------------------------------------------------------------------------------------------
% It is an example which *does* use the .bib file (from which the .bbl file
% is produced).
% REMEMBER HOWEVER: After having produced the .bbl file,
% and prior to final submission,
% you need to 'insert'  your .bbl file into your source .tex file so as to provide
% ONE 'self-contained' source file.
%
% Questions regarding SIGS should be sent to
% Adrienne Griscti ---> griscti@acm.org
%
% Questions/suggestions regarding the guidelines, .tex and .cls files, etc. to
% Gerald Murray ---> murray@hq.acm.org
%
% For tracking purposes - this is V3.1SP - APRIL 2009

\documentclass{acm_proc_article-sp}

\usepackage{auto-pst-pdf}

\begin{document}

\title{Robot-Induced Mimicry in Humans}
%
% You need the command \numberofauthors to handle the 'placement
% and alignment' of the authors beneath the title.
%
% For aesthetic reasons, we recommend 'three authors at a time'
% i.e. three 'name/affiliation blocks' be placed beneath the title.
%
% NOTE: You are NOT restricted in how many 'rows' of
% "name/affiliations" may appear. We just ask that you restrict
% the number of 'columns' to three.
%
% Because of the available 'opening page real-estate'
% we ask you to refrain from putting more than six authors
% (two rows with three columns) beneath the article title.
% More than six makes the first-page appear very cluttered indeed.
%
% Use the \alignauthor commands to handle the names
% and affiliations for an 'aesthetic maximum' of six authors.
% Add names, affiliations, addresses for
% the seventh etc. author(s) as the argument for the
% \additionalauthors command.
% These 'additional authors' will be output/set for you
% without further effort on your part as the last section in
% the body of your article BEFORE References or any Appendices.

\numberofauthors{5} 
\author{
% You can go ahead and credit any number of authors here,
% e.g. one 'row of three' or two rows (consisting of one row of three
% and a second row of one, two or three).
%
% The command \alignauthor (no curly braces needed) should
% precede each author name, affiliation/snail-mail address and
% e-mail address. Additionally, tag each line of
% affiliation/address with \affaddr, and tag the
% e-mail address with \email.
%
% 1st. author
\alignauthor
Apurv Suman\\
       \affaddr{Department of Computer Science}\\
       \affaddr{Yale University}\\
       \affaddr{New Haven, Connecticut 06520}\\
       \email{apurv.suman@yale.edu}
% 2nd. author
\alignauthor
Rebecca Marvin\\
       \affaddr{Department of Computer Science}\\
       \affaddr{Yale University}\\
       \affaddr{New Haven, Connecticut 06520}\\
       \email{rebecca.marvin@yale.edu}
\and 
% 3rd. author
\alignauthor Elena Corina Grigore\\
      \affaddr{Department of Computer Science}\\
       \affaddr{Yale University}\\
       \affaddr{New Haven, Connecticut 06520}\\
       \email{elena.corina.grigore@yale.edu}
% 4th. author 
\alignauthor Henny Admoni\\
      \affaddr{Department of Computer Science}\\
       \affaddr{Yale University}\\
       \affaddr{New Haven, Connecticut 06520}\\
       \email{henny.admoni@yale.edu} 
% 5th. author
\alignauthor Brian Scasselatti\\
      \affaddr{Department of Computer Science}\\
       \affaddr{Yale University}\\
       \affaddr{New Haven, Connecticut 06520}\\
       \email{brian.scasselatti@yale.edu}
}
\date{03 October 2014}
% Just remember to make sure that the TOTAL number of authors
% is the number that will appear on the first page PLUS the
% number that will appear in the \additionalauthors section.

\maketitle
\begin{abstract}
As robots become more and more integrated with our lives, particularly beyond a functional role, there is a greater need to understand human-robot interaction at a deeper level. One such area in need of further exploration is mimicry between humans and robots, particularly in the direction of humans mimicking robots. This study explores human-robot interaction in terms of behavioral mimicry - the automatic imitation of gestures, postures, mannerisms, and other motor movements. We conducted an experiment looking at how much humans mimicked a robot during the task of describing paintings. We find that humans do mimic a robot more after observing a robot perform an action. We also find that humans already performing a behavior actually perform that behavior less after seeing the robot perform it.
\end{abstract}

\keywords{human-robot interaction, mimicry, chameleon effect} % NOT required for Proceedings

\section{Introduction}
We now live in a world where robots have become a part of our daily lives. Whether it is Siri in your pocket or the recently released Jibo in your home, robots have moved from a purely functional space to an interactive and social space. With this new movement, there is a pressing need to understand the social effects and impact of robots, particularly humanoid or anthropomorphized robots. Understanding the social effects of a robot will provide us with new information on how to better design robots for human interaction, how to structure human environments for robot presences, and how to anticipate the impact robots will have on our society moving forward. 

In trying to understand the social effects of robots, robotics must look to the field of psychology. Psychology can provide us a framework and a baseline to assess human-robot interaction. This is done by taking social constructs and effects from human-human interaction and mapping them to robots. This space of human-human effects mapped to human-robot interaction is still largely unexplored. In this study, we hope to elucidate a part of this space: automatic mimicry.

Behavioral mimicry - the automatic imitation of gestures, postures, mannerisms, and other motor movements, is pervasive in human interactions \cite{chartrand2013antecedents}. \textit{Automatic} mimicry in particular is often unconscious and unintentional. For example, studies have shown participants mimicking confederates tapping their feet or touching their faces during interactions, even without realizing they were doing so \cite{chartrand1999chameleon}. Furthermore, research has also shown that participants found confederates more likeable when the confederates mimicked the postures of the participants. There are several reasons why such a behavior exists. From a neurological standpoint, there exist certain parts of the brain responsible for this mimicry, known as mirror neurons or the ``mirror system'' in humans, first highlighted in macaques in 1992 at the University of Parma \cite{ehrenfeld2011reflections}. In fact an actual perception-behavior link exists behind automatic mimicry. This means that actions seen can trigger the mirror system in humans \cite{chartrand1999chameleon}. Beyond the neurology, there is a social component and possible evolutionary explanation for mimicry. Mimicry has been shown to create liking, empathy, and affiliation between interactants and act as the ``social glue'' that brings people together and bonds them \cite{chartrand2013antecedents}, \cite{lakin2003chameleon}. Also, research suggests that mimicry serves a prosocial communicative purpose \cite{bavelas1986show}, \cite{chartrand2013antecedents}.

We find several reasons as to why automatic mimicry should be studied. In the broadest sense, automatic mimicry is a thinly explored area of human-robot interaction that has a parallel in human-human interaction. Given that psychological research on mimicry has highlighted questions about mimicking the ``right'' people (i.e. ingroups vs. outgroups) \cite{bourgeois2008impact}, \cite{chartrand2013antecedents}, \cite{kavanagh2011s}, \cite{yabar2006implicit}, mimicry could highlight how humans view robots as social actors. Also, it has been observed that mimickers, and not just mimickees, have smoother interactions, smoother negotiations, more interpersonal trust, and more likability with their partners \cite{maddux2008chameleons}, \cite{stel2010mimicry}, \cite{swaab2011early}, so finding out if automatic mimicry exists in humans could give us information on how to shape interactions between humans and robots. From a practical standpoint, if robots can induce mimicry in humans, it provides us insight in to how a robot may induce behavior in humans within the real world as well. One can foresee possible concerns of a child or infant mimicking a robot in the home. One could also foresee issues regarding individuals mimicking a robot in public, perceiving the robot's actions as a sign of permissibility. All these new questions, however, rest on the assumption that a robot could in fact elicit mimicry in humans, an assumption we hope to probe in this study. Observing no mimicry or some kind of adverse social effect would also raise questions about why the perception-behavior link in human-human interaction does not extend to human-robot interaction.

Lastly, automated mimicry of physical behaviors is the first step in the larger phenomena of social contagion \cite{chartrand2013antecedents}, so understanding how human-robot interaction works for automated mimicry can act as a stepping stone to social contagion with human-robot interaction.

We hypothesize that:\\
\textbf{H1}	People who do not spontaneously exhibit a behavior will perform that behavior more after seeing a robot perform it.\\
\textbf{H2} People who do spontaneously exhibit a behavior will perform that behavior more after seeing a robot perform it.

We define spontaneous exhibition of a behavior to be performing a specified behavior for any period of time prior to observing the robot perform said behavior.

Our results support H1 and reject H2. Non-spontaneous participants mimicked the robot after observing it perform the specified behavior. Surprisingly, spontaneous participants mimicked the robot less after observing the robot it perform the specified behavior.

This work suggests that robots can induce non-emotional and non-communicative mimicry in humans. It also highlights that there exist human expectations certain expectations for mimicry with robots for humans.

\section{Relevant Work}
There is evidence in research that suggests a robot could elicit automatic mimicry in humans. Oberman demonstrated through EEG that activation of the mirror neuron system in humans can occur through the perception of robot behavior, even without objects. This is significant as it informs us that the mirror neuron system and the perception-behavior link in humans is not uniquely limited to perceiving and reacting to human actions \cite{oberman2007eeg}. 

Bailenson successfully demonstrated that liking, rapport, and affiliation can be increased with mimicry even with a digital agent, which they showed using a virtual agent on a computer screen mimicking the head posture of the participants. This highlights one direction of the mimicry effect, in which a participant being mimicked has a more positive experience with a mimicker. The existence of this direction with a non-human agent suggests the possibility of the opposite direction, a human mimicking a non-human agent \cite{bailenson2005digital}. 

Riek conducted a study hoping to observe improved likability with an embodied robot (resembling an ape) mimicking head posture of a participant. Their study identified problems with assessing human-robot interactions using a survey and the difficulties of capturing and mimicking behaviors between humans and robots. Their work guided us in our planning of behaviors \cite{riek2010my}. Riek did find some support for more satisfactory interactions when facial expressions were mimicked by the same ape-like robot, although the findings were preliminary and in a pilot \cite{riek2008real}.

Hofree most significantly found that humans can spontaneously match facial expressions of an embodied android present in the room. Their study suggested that the salience of mimicry depended on how human-like the android presented is. This finding pushed our study to focus on a robot that minimized human-like features and emotions, allowing us to test on a more basic set of behaviors, devoid of expressions of emotions such as anger or happiness \cite{hofree2014bridging}. 

\section{Methods} 
In order to focus on \textit{automatic} mimicry, participants were given a task to complete while interacting the robot. Given the lack of research on robots inducing mimicry in humans, our study borrowed heavily from Chartrand \& Bargh's experimental design \cite{chartrand1999chameleon}. 

Participants alternated describing paintings with a Nao robot. Halfway through the trial, the Nao would assume a posture (either putting its hands behind its back or putting its hands on its hips) while continuing with the description of paintings. We measured the time the participant assumed the posture both before and after the Nao assumed the posture, making this a within-subjects study. The ``before'' period acted as the control while the ``after'' period acted as the observed variable.

This whole process was repeated with a different Nao for the same participant. The different Nao was brought in by the experimenter immediately after the conclusion of the first session. During the second session, the Nao performed whichever behavior the first Nao did not (either putting its hands behind its back or putting its hands on its hips). This gave us a larger data set of behaviors to observe and also helped us to control for participants who had existing tendencies to do one behavior or another.

Using two separate Naos came from the use of two different confederates in  the Chartrand \& Bargh study, which also used two different behaviors (touching the face and tapping the feet) \cite{chartrand1999chameleon}. The use of continuous postures rather than a discrete behavior was largely due to limitations of the robot, but we had little to reason to believe postures would not be effective behaviors given the extensive use of postures in mimicry research \cite{chartrand2013antecedents}. 

Our study aimed to minimize features that would increase the likelihood of mimicry such as goal to affiliate, which has been shown to increase mimicry \cite{chartrand2013antecedents}, \cite{drury2006effects}, \cite{lakin2003using}. The Nao's descriptions of the paintings were kept to be as simple as possible, with little to no emotion or interpretation \cite{hofree2014bridging}. The Nao also made no acknowledgement of the participants or their descriptions and the behaviors of hands behind back and hands on hips were chosen to minimize postural communication . 

\subsection{Robot Platform}
Nao is a 58-cm tall humanoid robot. Nao has 25 degrees of freedom, 2 cameras, 4 microphones, speakers, touch sensors, and an inertial measurement unit. For the study, Nao's legs were employed for standing up, Nao's arms were used to assume 1 of the 2 aforementioned postures, Nao's touch senors were used to start a script of behaviors, and Nao's speakers were used to voice the painting descriptions \cite{naodocumentation}. 

Nao is designed by Aldebaran on the Naoqi operating system. Python was used to program the Nao for the study. 

\subsection{Procedure}
Participants were first asked to fill out consent and video release forms. Participants were then randomly assigned to a group that saw hands behind back or hands on hips during the first session (with the other behavior occurring in the second session). Participants were then brought in to a closed 420cm x 300cm room. Participants faced the Nao at a distance of 180cm. The Nao stood on a platform raised 75cm off the ground. Next to the Nao stood next to a 68cm Apple iMac Display which ran Python scripts for the Nao's behaviors through a local area network connected to the Nao's head and displayed a PowerPoint presentation on Keynote. A GoPro was located at the back of the room and aimed at the participant's back. A camcorder was placed in the corner of the room facing the participant.

The participant was asked to read the first slide of directions while the experimenter turned on the cameras. Next, the experimenter started the PowerPoint and tapped the Nao's head sensor to start the Nao's Python scripts. The PowerPoint slides and scripts were synchronized. The Nao would turn its head to ``see'' the painting, turn back to the participant, and describe a painting (descriptions were pre-scripted) for 1 minute after which the participant was notified on-screen to do so as well. This continued for 3 paintings total. At this point, the Nao performed the assigned behavior for session 1 and maintained that posture for 3 more paintings. After 6 total paintings, the Nao returned to a crouch position and the experimenter replaced the Nao with a new one. The same process was repeated with 6 more paintings, except the other behavior was performed in session 2. 

After both sessions were completed, the participant completed a survey on Qualtrics, received \$5, and left.

\subsection{Data Collection}
Video of the participants were collected through a camcorder that captured the front of the participant and a GoPro that captured the back of the participant (this was necessary in order to validate any behaviors the participants portrayed behind their backs).

The videos were coded using ELAN 4.7.2. Both behaviors and variations of the behaviors were coded for. Participants also filled out a survey comprising of Likert Scale questions on intelligence and likability, short answer questions on what they liked/noticed about the trial, and demographic questions. Participants comprised of 49 Yale Undergraduates of which 43 were used in the final data analysis (6 participants did not qualify for the study or experienced a technical problem, such as losing internet connection, during the trial).

\section{Results}
This experiment yielded quantitative results from video coding of the recordings of participants and from self-reporting through a survey of Likert scale and short-answer questions. 

\textbf{Video Coding} We coded a participant putting hands on hips with 4 different designations and a participant putting hands behind back with 3 different designations. The different designations ensured we were able to cover all variations of the behaviors in the event they were displayed. For hands on hips the designations were two hands on hips, one hand on hip, hands in pockets, and hands in belt loops. We ultimately did not use hands in pockets because it did not accurately resemble the Nao's behavior, and we ultimately did not use hands in belt loops because it was very rarely performed by participants. For hands behind back the designations were two hands behind back, one hand behind back, and hands in back pockets. We ultimately did not use hands in bock pockets because it was very rarely performed by participants.
	
\textbf{Strict vs. Loose} For our analysis we broke both behaviors in to a strict and loose definition. For hands on hips, the strict definition matches the Nao's behavior of putting two hands on hips. The loose definition is a superset of this, with the participant exhibiting \textit{at least} one hand on hip. For hands behind back, the strict definition matches the Nao's behavior of putting two hands behind back. The loose definition is a superset of this, with the participant exhibiting \textit{at least} one hand behind back. Having a strict and loose interpretation allowed us to take into account participants who partially performed the behavior in our analysis. 

\textbf{Spontaneity} Participants were separated in to two different populations with different distributions. Those who performed the respective behavior at any point before the Nao did were considered spontaneous while those who did not were considered non-spontaneous.

\textbf{Non-Spontaneous Mimicry} The central question of this study is whether or not a robot can induce mimicry in humans. By definition, participants who did not spontaneously perform hands on hips or hands behind back had an average time performing those behaviors of 0 milliseconds. 

For the strict definition of hands on hips, non-spontaneous participants performed the behavior an average of 9280.71 milliseconds more after the Nao put its hands on its hips. A paired one-tailed t-test (n=34) found this result to be statistically significant with a p-value of 0.008. For the loose definition of hands on hips, non-spontaneous participants performed the behavior an average of 11592.13 milliseconds more after the Nao put its hands on its hips. A paired one-tailed t-test (n=32) found this result to be statistically significant with a p-value of 0.008. 

For the strict definition of hands behind back, non-spontaneous participants performed the behavior an average of 13420.10 milliseconds more after the Nao put its hands behind its back. A paired one-tailed t-test (n=30) found this result to be marginally significant with a p-value of 0.066. For the loose definition of hands behind back, non-spontaneous participants performed the behavior an average of 22292.48 milliseconds more after the Nao put its hands behind its back. A paired one-tailed t-test (n=29) found this result to be statistically significant with a p-value of 0.035.

\textbf{Spontaneous Mimicry} Our study identified that participants who spontaneously performed hands on hips or hands behind back independent of the robot doing so as a different population who would be impacted differently by the Nao performing a behavior they already were. This was confirmed in our survey data by numerous responses stating that the participant thought the robot was mimicking the participant's behavior. Because it is very difficult to pre-select participants who we know will spontaneously perform the behaviors, we had to rely on those who did within our overall population. As a result, we recognize the sample sizes are smaller and that are conclusions are weaker for people who do spontaneously perform a behavior.

For the strict definition of hands on hips, spontaneous participants performed the behavior an average of 90381 milliseconds before the Nao put its hand on its hips and an average of 42047 milliseconds after the Nao put its hand on its hips. A one-tailed paired t-test (n=9) found the difference of -48334 milliseconds to be statistically significant with a p-value of 0.025580944. For the loose definition of hands on hips, spontaneous participants performed the behavior an average of 80540.36 milliseconds before the Nao put its hand on its hips and an average of 39655.36 milliseconds after the Nao put its hand on its hips. A one-tailed paired t-test (n=11) found the difference of -40885 milliseconds to be statistically significant with a p-value of 0.030384716.

For the strict definition of hands behind back, spontaneous participants performed the behavior an average of 116665.08 milliseconds before the Nao put its hands behind its back and an average of 49043.77 milliseconds after the Nao put its hands behind its back. A one-tailed paired t-test (n=13) found the difference of -67621.30769 milliseconds to be statistically significant with a p-value of 0.040191878. For the loose definition of hands behind back, spontaneous participants performed the behavior an average of 135802.93 milliseconds before the Nao put its hands behind its back and an average of 73551.36 milliseconds after the Nao put its hands behind its back. A one-tailed paired t-test (n=14) found the difference of -62251.57 milliseconds to be marginally significant with a p-value of 0.051.

\textbf{Survey Results} Beyond providing us insight for our inferences in our discussion, we found no significant result for likability, intelligence, gender, or race and the performance of behaviors before or after the Nao performed them. Our lack of results here can be partially explained by lack of sufficient sample size for questions such as demographic analysis.

\textbf{Statistical Methods} Our determinations for statistical significance used the following guidelines and justifications. P-values less than 0.05 were deemed statistically significant while p-values between 0.05 and 0.1 were deemed marginally significant. Given that the experiment was run as a within-subjects study, a paired t-test was used. Finally, we used one-tailed t-tests because of our division of spontaneous and non-spontaneous participants based on their performance of a behavior being 0 or positive before the Nao performed the respective behavior. As a result, the direction for non-spontaneous participants performing behaviors after the Nao did so drew from a population mean that was positive while the direction for spontaneous participants performing behaviors after the Nao did so drew from a population mean that was negative.

\section{Discussion}
Our results yield two main findings about inducing mimicry in humans through robots.
RESULT 1. \textit{Humans who do not spontaneously demonstrate a behavior prior to observing a robot do so perform that behavior more after observing a robot perform it.}
RESULT 2. \textit{Humans who spontaneously demonstrate a behavior prior to observing a robot do so perform that behavior less after observing a robot perform it.}

\textbf{Humans Mimic Robots} Our results support our first hypothesis H1. For both the strict and loose definitions of hands on hips, participants significantly put their hands on their hips more after seeing the Nao do so. This presents very strong evidence for a robot's ability to induce mimicry. For the hands behind back behavior, the strict definition was only marginally significant while the loose definition was statistically significant. This makes sense as the loose definition focuses on the population who put neither hand behind the back. This means no part of this group even partially performed the behavior spontaneously. As far as why the strict definition of hands behind back was only marginally significant while the strict definition of hands on hips was statistically significant, the difference can significance can possibly be explained by the nature of the behaviors themselves. Hands on hips is a more stable behavior in that occurrences of it are longer in duration than the sometimes short burst of hands behind back. This is best captured by the sample of size of strict hands on hips vs. strict hands behind back (34 vs. 30). Essentially, there are less participants who by chance or very quickly/casually perform a hands on hips behavior spontaneously. This is especially true with one hand behind back, which is why the loose definition helps to bring hands behind back into statistical significance (the loose definition separates the populations of spontaneous vs. non-spontaneous more thoroughly).

\textbf{Spontaneous Performers Mimic Less} Our results reject our second hypothesis H2. This was a surprising yet exciting result of our study. Participants who spontaneously performed hands on hips or hands behind back prior to the Nao doing so actually performed that behavior \textit{less} after the Nao did. While we cannot conclude why this would happen we can make inferences, especially with the help of our survey responses. Several responses noted spontaneous participants saying that when they saw the Nao first put its hands on its hips or put its hands behind its back, they thought the Nao was mimicking their behavior. Such a thought process makes sense when we recall that the spontaneous participants had performed that behavior already at some point before ever seeing the Nao do so. The fact several participants were trying to figure out the purpose of the study only further pushed them to attribute the Nao's behavior to the cause of mimicry (albeit in the wrong direction). This stands in contrast to several non-spontaneous participant responses who correctly identified the purpose of the study and did not confuse the direction of mimicry since they had never performed it up until that point. With this idea in hand that participants thought they were being mimicked, we can go back to psychology literature to look for explanations. Mimicry research in psychology has shown that mimicry can lead to socially cold feelings or the feeling that something is ``off''. In particular, an inappropriate amount of mimicry arouses suspicion in the party being mimicked \cite{bargh2012substitutability}, \cite{leander2012you}, \cite{stel2010mimicking}, \cite{zhong2008cold}. This can possibly explain why there is a decrease in performance of hands on hips or hands behind back for spontaneous participants. Another possible explanation is the possibility that humans view the Nao as members of a social outgroup. It is possible that by viewing the Nao as a member of an outgroup that perceived mimicry by the Nao is inappropriate \cite{kavanagh2011s}; however, given that the non-spontaneous group does show some mimicry it is unclear what conclusions can be drawn about social ingroup/outgroup status of the Nao and robots more generally.

There are several caveats to consider with this theory, however. Previous studies showed improved relations between a human and a digital avatar when head posture was mimicked \cite{bailenson2005digital}. In that study, however, the mimicking was done on a delay and participants were not aware of the mimicry \cite{bailenson2005digital}. In our study, participants saw what they thought was an explicit attempt at mimicry. The explicit attempt fits more appropriately with the psychology research that discusses problems with ``inappropriate'' amounts of mimicry.

Within the surprising finding for spontaneous performers of the behavior is that the loose definition saw smaller decreases than the strict definition. For hands behind back, the loose definition had such a smaller decrease that it even moved the loose definition for spontaneous participants from statistically significant to marginally significant. This can possibly be explained by the fact that our cutoff for spontaneous vs. non-spontaneous was 0. This means that participants who barely performed the behaviors before the Nao did, even for only 1000 milliseconds, were counted in the spontaneous group. These participants had such a low ``before'' time that they could easily have a higher ``after'' time. This could happen just by chance (particularly for hands behind back which had more small burst of the behavior) or because the participants who performed the behavior for small periods in the ``before'' period did not take the Nao performing the behavior as an attempt at mimicry, silencing the concern of inappropriate mimicry. Fundamentally, this problem is more pronounced given the small sample size for the spontaneous group.

\textit{Implications}
In light of our first finding we can conclude that robots can induce mimicry in humans. Ultimately, this may raise more questions than it answers, but that does create a wealth of possibilities for further research questions. Does the salience of mimicry in human-robot interaction move in the same patterns as it does in humans? For example, does having a goal to affiliate or similarity between partners induce greater mimicry in human-robot interaction as it does in humans \cite{chartrand2013antecedents}. Our second finding also raises concerns about building mimicry into human-robot interaction, especially in terms of having robots mimic humans. Perhaps there are certain prerequisites that must be fulfilled in a human-robot partnership before mimicry is considered appropriate. Lastly, this study raises issues for design of robots and their impact on humans around them. Just as we are wary of how other humans may mimic our actions, we must be wary of how robots may induct humans to mimic tasks, particularly if they are tasks we design robots to do specifically because we don't want humans doing them.

%ACKNOWLEDGMENTS are optional
\section{Acknowledgments}
The authors would like to thank Rachel Protacio and Jessica Yang for their help in the piloting of this study. They would also like to thank John Bargh of Yale University for his foundational work in the Chameleon Effect and for personally helping with the background research for human-human mimicry.

%
% The following two commands are all you need in the
% initial runs of your .tex file to
% produce the bibliography for the citations in your paper.
\bibliographystyle{abbrv}
\bibliography{sigproc}  % sigproc.bib is the name of the Bibliography in this case
% You must have a proper ".bib" file
%  and remember to run:
% latex bibtex latex latex
% to resolve all references
%
% ACM needs 'a single self-contained file'!
\end{document}
